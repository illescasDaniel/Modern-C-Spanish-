\documentclass{beamer}

\usetheme{metropolis}
\usepackage{listings} % Código
%\usepackage[utf8]{inputenc}
\usepackage[normalem]{ulem} % Tachar texto (con \sout)
\usepackage[spanish]{babel}
\usepackage{hyperref}
\usepackage{color}

\definecolor{dkgreen}{rgb}{0,0.52,0}
\definecolor{gray}{rgb}{0.5,0.5,0.5}
\definecolor{mauve}{rgb}{0.4,0,0.7}
\definecolor{customBlue}{rgb}{0.1,0.1,0.65}
\definecolor{customOrange}{rgb}{0.88,0.53,0.13}

% Cambiar tipo letra
%\setsansfont{Arial}

\usepackage{fontspec}
\setmonofont[
  Contextuals={Alternate}
]{Fira Code}

\lstset{frame=tb,
  language=c++,
  aboveskip=5mm,
  belowskip=5mm,
  showstringspaces=false,
  columns=flexible,
  basicstyle={\scriptsize\ttfamily},
  numbers=left, 
  numberstyle=\tiny, 
  stepnumber=1, 
  numbersep=8pt,
  numberstyle=\tiny\color{gray},
  keywordstyle=\color{customBlue},
  commentstyle=\color{dkgreen},
  stringstyle=\color{mauve},
  breaklines=true,
  breakatwhitespace=true,
  tabsize=4,
  gobble=16
}

\lstset{literate=
    {->}{{{\texttt{-> }}}}1
    {<-}{{{\texttt{<- }}}}1    
    {<->}{{{\texttt{<-> }}}}1
}

%\lstset{literate=%
%    *{0}{{{\color{customOrange}0}}}1
%    {1}{{{\color{customOrange}1}}}1
%    {2}{{{\color{customOrange}2}}}1
%    {3}{{{\color{customOrange}3}}}1
%    {4}{{{\color{customOrange}4}}}1
%    {5}{{{\color{customOrange}5}}}1
%    {6}{{{\color{customOrange}6}}}1
%    {7}{{{\color{customOrange}7}}}1
%    {8}{{{\color{customOrange}8}}}1
%    {9}{{{\color{customOrange}9}}}1
%}

\newcommand{\normalSizeItem}[1] {
  \normalsize{\item #1}
}

\newcommand{\newFrameWithoutIndex}[1]{
	\begin{frame}
		#1
		\thispagestyle{empty}
	\end{frame}
}

\newcommand{\newSectionWithoutIndex}[1]{
	\newFrameWithoutIndex{\section{#1}}
}

\newcommand{\urlItem}[1]{
	\footnotesize{\item {\url {#1}}}
}

\newcommand{\tinyUrlItem}[1]{
	\tiny{\item {\url {#1}}}
}
  
\newcommand{\smallCite}[1]{
	\begin{small}
		\cite{#1}	
	\end{small}
}

\title{Modern C++}
\author{Daniel Illescas Romero}
\date{\today}
\institute{Universidad de Granada [UGR]}

\setcounter{tocdepth}{1} % 2 para mostrar subsecciones
\setcounter{secnumdepth}{2}
\setbeamerfont{subsection in toc}{size=\tiny}

\begin{document}

	\newFrameWithoutIndex{\maketitle}
	
	\begin{frame}{Índice}		
		\setbeamertemplate{section in toc}[sections numbered]
		\tableofcontents
	\end{frame}

	\newSectionWithoutIndex{Estilo y buenas prácticas}
  
		\begin{frame}[fragile]{Nombrado de variables}	
			\subsection{Nombrado de variables}
			\begin{itemize}
			
				\normalSizeItem {Mala idea}
				\begin{lstlisting}
				int main() {
					int a, b;
					double c;
				}
				\end{lstlisting}
				
				\normalSizeItem {Buena idea}
				\begin{lstlisting}
				int main() {
					unsigned int age = 20;
					int points = 0;
					float height = 1.7;
				}
				\end{lstlisting}
				
			\end{itemize}
		\end{frame}
			 
		\begin{frame}[fragile]{Nombrado de variables}	
			\begin{itemize}
			
				\normalSizeItem{Mejor idea \textbf{[C++11]}}
				\begin{lstlisting}
				int main() {
					uint8_t age{10};
					int64_t points{}; 
					float height{};
				}
				\end{lstlisting}
				
				\normalSizeItem{Friki! \textbf{[C++11]}}
				\begin{lstlisting}
				int main() {
					auto age = uint8_t{20};
					auto points = int64_t{0};
					auto height = float{5.67};
				}
				\end{lstlisting}
				
			\end{itemize}
		\end{frame}
		
		\begin{frame}[fragile]{Estilos de programación}	
			\subsection{Estilos de programación}
			\begin{itemize}
			
				\normalSizeItem{Nombres variables \smallCite{namingConvention}}
				\begin{lstlisting}
				// Palabras separadas por un delimitador
				float daniel_height = 1.7;
				// CamelCase (separadas por mayúsculas)
				float johnHeight = 1.7;
				\end{lstlisting}
				
				\normalSizeItem{Llaves (Nota: ¡usadlas siempre!) \smallCite{indentationStyle}}
				\begin{lstlisting}
				// K&R (Kernighan and Ritchie)
				if () {
				
				}
				// Allman
				if ()
				{
				
				}
				\end{lstlisting}
				
			\end{itemize}
		\end{frame}
		
		\begin{frame}[fragile]{Macros y variables globales}	
			\subsection{Evitar macros y variables globales}
			\begin{itemize}
			
				\normalSizeItem{Evita las macros}
				\begin{lstlisting}
				// ¿Tipo?, ¿Scope?
				// No es constante y puede des-definirse
				#define PI 3.14159
				#define SIZE 10
				
				int main() {
					
					// float PI = 10; -> Sería: float 3.14159 = 10; ¿?
					
					int numbers[SIZE];
					
					#undef SIZE
					
					// Error: "SIZE" no está declarado...
					cout << SIZE << endl;
				}
				\end{lstlisting}
				
			\end{itemize}
		\end{frame}
		
		\begin{frame}[fragile]{Macros y variables globales}	
			\begin{itemize}
			
				\normalSizeItem{Procura NO usar variables globales}
				\begin{lstlisting}[basicstyle={\tiny\ttfamily}]
				#include <vector>
				#include <array>
				
				// Da lugar a problemas... MALA IDEA
				// ERROR por ambigüedad con: std::__1::size
				int size = 20; // ¿Podría ser negativo un tamaño también?
				
				namespace utils { // Esto sería una alternativa segura
					constexpr size_t arraySize = 20;
				}
				
				// ¿Cómo paso un array entonces? Así NO sabes tamaño
				void doSomething(int arr[]) {}
				
				void doSomething(int arr[size]) {} // NO...
				
				// Opción válida
				void doSomething(int* arr, const size_t arrSize) {}
				
				// Si sabemos tamaño
				void doSomething(std::array<int,20> arr) {}
				
				// Si NO sabemos tamaño
				void doSomething(std::vector<int> arr) {}
				\end{lstlisting}
				
			\end{itemize}
		\end{frame}
		
		\begin{frame}[fragile]{Dependencia de plataformas}	
			\subsection{Dependencia de plataformas}		
			\begin{itemize}

				\normalSizeItem { El uso de \texttt{system} puede llevar a errores}
				\begin{lstlisting}
				// No funciona en la línea de comandos de Windows
				std::system("ls -l > test.txt");
				
				// Solo funciona en Windows :/
				std::system("pause"); 
				\end{lstlisting}
				
				\item { 
					
					Algunas soluciones: 
								
					\begin{itemize}
						\item {
							Para listar ficheros en cualquier plataforma podemos usar:
							 \begin{verbatim}
							 	std::filesystem::recursive_directory_iterator
							 \end{verbatim}
						}
						\item {
							Para pausar la ejecución del programa: \texttt{cin.ignore();}
						}
					\end{itemize}
				}
			\end{itemize}
		\end{frame}
		
		\begin{frame}[fragile]{Dependencia de plataformas}	
			\subsection{Dependencia de plataformas}		
			\begin{itemize}
				\normalSizeItem { Si se utlizan librerías antiguas y dependientes no tenemos código \textit{portable} (ni bonito a veces...) \smallCite{posixThreads}}
						\begin{lstlisting}[basicstyle={\tiny\ttfamily}]
						#include <pthread.h>
		
						void* print_message_function(void* ptr) {
							char *message;
							message = (char *) ptr;
							printf("%s \n", message);
						}
						
						int main() {
							int iret1 = pthread_create(&thread1, NULL, print_message_function, (void*) message1);
						}
		
						\end{lstlisting}
			\end{itemize}
		\end{frame}
		
		\begin{frame}[fragile]{Dependencia de plataformas}	
			\begin{itemize}

				\normalSizeItem { Procura usar librerías propias del lenguaje y/o actualizadas }
				\begin{lstlisting}[basicstyle={\tiny\ttfamily}]
				#include <thread>
				
				void doSomething(int number) { ... }
				void doSomething2(int& number) { ... }
				
				int main() {
				
					int number = 0;
					
					// paso por valor
					std::thread thread1(doSomething, number + 1); 
					
					// paso por referencia
					std::thread thread2(doSomething2, std::ref(number)); 
					
					// thread3 está corriendo "doSomething2()", thread2 ya 'no es un thread'
					std::thread thread3(std::move(thread2)); 
					
					thread1.join();
					thread3.join();					
				}
				
				\end{lstlisting}
				
			\end{itemize}
		\end{frame}
		
		\begin{frame}[fragile]{Semantic Versioning \smallCite{semanticVersioning} - Control de Versiones}	
			
			\textbf{Nomenclatura:} MAJOR.MINOR.PATCH
			\newline\newline
			Ejemplo: \textbf{1.3.4}
		
			Incrementamos la versión cuando:
			\begin{itemize}

				\normalSizeItem { \textbf{MAJOR}: se hacen cambios incompatibles de API}
				\normalSizeItem { \textbf{MINOR}: al añadir funcionalidades retrocompatibles }
				\normalSizeItem { \textbf{PATH}: correcciones de errores }
			\end{itemize}
		\end{frame}
		
		\begin{frame}[fragile]{Semantic Versioning - Control de Versiones}	
			
			Forma de trabajar:
			\begin{itemize}
				\item Empezar en la versión \textbf{0.1}
				\item Ir incrementando la versión \textbf{0.x.y }mientras sea SW en desarrollo
				\item Usar la \textbf{v1.0.0} para producción
				\item No te pases, un número de versión muy grande no es legible
			\end{itemize}
		\end{frame}
		
		\newSectionWithoutIndex{Nuevo en C++}	
		
		\begin{frame}[fragile]{\texttt{auto}, inferencia de tipos -\smallCite{cppReference}}\subsection{\texttt{auto}, inferencia de tipos}	
			\begin{itemize}
			
				\normalSizeItem{Tipos básicos o simples}
				\begin{lstlisting}
				auto number = 2; // int
				auto width = 0.5; // double
				auto name = "daniel"; // const char*
				auto name = "daniel"s; // string
				\end{lstlisting}
				
				\normalSizeItem{Tipos más complejos}
				\begin{lstlisting}[basicstyle={\tiny\ttfamily}]
				vector<int> points{1,2,3};
				
				for (vector<int>::iterator it = points.begin(); it != points.end(); ++it) {
					cout << *it << endl;
				}
				for (auto it = points.begin(); it != points.end(); ++it) {
					cout << *it << endl;
				}
				\end{lstlisting}
				
			\end{itemize}
		\end{frame}
		
		\begin{frame}[fragile]{\texttt{auto}, inferencia de tipos}	
			\begin{itemize}
			
				\normalSizeItem{En funciones con tipos... un poco largos}
				\begin{lstlisting}[basicstyle={\tiny\ttfamily}]
				std::chrono::time_point<std::chrono::high_resolution_clock> now() {
					return std::chrono::high_resolution_clock::now();
				}
				
				auto now() {
					return std::chrono::high_resolution_clock::now();
				}
				\end{lstlisting}
								
				\normalSizeItem { Otras formas de usarlo }
				\begin{lstlisting}[basicstyle={\tiny\ttfamily}]
				auto add(int value, double value2) -> decltype(value + value2) {
					return value + value2;
				}
				auto calculateWeight(...) -> double { ... }
				\end{lstlisting}
				
			\end{itemize}
		\end{frame}
		
		\begin{frame}[fragile]{Enteros de longitud fja [C++11] y \textit{brace initializer}}	
			\subsection{Enteros de longitud fja [C++11] y "brace initializer"}
			\begin{itemize}
			
				\normalSizeItem{Tipos principales}
				\begin{lstlisting}[basicstyle={\tiny\ttfamily}]
				#include <cstdint>

				int8_t, uint8_t  // Entero 8 bits con signo, sin signo
				int16_t, uint16_t // 16 bits
				int32_t, uint32_t  // 32 bits
				int64_t, uint64_t // 64 bits
				intmax_t, uintmax_t // Mayor capacidad (normalmente 64 bits)
				size_t // Para representar capacidades [Entero 64 bits sin signo]
				\end{lstlisting}
								
				\normalSizeItem { Inicializar correctamente tipos }
				\begin{lstlisting}
				uint64_t bigStuff {643456787654};
				int32_t things = {3500000000};
				// uint8_t age {-20}; // Error al compilar
				things += {2000000000};
				// things = 1205032704 :(
				// no se puede evitar el overflow...
				\end{lstlisting}
				
			\end{itemize}
		\end{frame}
		
		\begin{frame}[fragile]{\sout{NULL}, \ttfamily{nullptr}}	
			\subsection{\sout{NULL}, \ttfamily{nullptr}}
			\begin{itemize}
			
				\normalSizeItem { ¡NULL da lugar a errores! }
				\begin{lstlisting}[basicstyle={\tiny\ttfamily}]
				void doSomething(int* ptrI) { ... }
				void doSomething(double* ptrD) { ... }
				void doSomething(std::nullptr_t nullPointer) { ... } 
				
				int main() {
				
					int* ptrI;
					double* ptrD;
				 
					doSomething(ptrI);
					doSomething(ptrD);
					doSomething(nullptr);  // ambiguo sin void f(nullptr_t)
					//doSomething(NULL);  // ambiguo: todas las funciones son candidatas
				}
				\end{lstlisting}
				
				\normalSizeItem { Asignar null}
				\begin{lstlisting}
				int* something = NULL; // antes C++11: NULL = 0
				double* foo = nullptr; // C++11: std::nullptr_t
				\end{lstlisting}
				
			\end{itemize}
		\end{frame}
		
		\begin{frame}[fragile]{Alternativa a \textit{raw pointers}. \ttfamily{unique\_ptr<int>}}	
			\subsection{Alternativa a raw pointers. \ttfamily{unique\_ptr<int>}}
			\begin{itemize}
			
				\normalSizeItem { Liberación de memoria }
				\begin{lstlisting}[basicstyle={\tiny\ttfamily}]
				#include <memory>
				
				using namespace std;
				
				int* giveMeAnInt() { 
					return new int{100};
				}
				
				unique_ptr<int> giveMeACoolInt() {
					return unique_ptr<int>(new int{100});
				}
				
				auto giveMeTheBestInt() {
					return make_unique<int>(100);
				}
				
				int main() {	
					// unique_ptr libera automáticamente memoria, int* no
					auto number = giveMeTheBestInt();
					cout << *number << endl;
					
					int* otherNumber = giveMeAnInt();
					delete otherNumber;
				}
				\end{lstlisting}				
			\end{itemize}
		\end{frame}
		
		\begin{frame}[fragile]{Alternativa a \textit{raw pointers}. \ttfamily{unique\_ptr<int>}}	
			\begin{itemize}
			
				\normalSizeItem { Liberación de memoria }
				\begin{lstlisting}
				#include <iostream>
				#include <memory>
				
				using namespace std;
				
				int main() {	
					
					constexpr size_t arrSize { 1000 };
					
					unique_ptr<int[]> numbers(new int[arrSize]);
					//[C++14] auto numbers = make_unique<int[]>(arrSize);
					
					fill(&numbers[0], &numbers[arrSize], 10);
					
					for (size_t i = 0; i < arrSize; i++) {
						cout << numbers[i] << endl;
					}
				}
				\end{lstlisting}				
			\end{itemize}
		\end{frame}
		
		\begin{frame}[fragile]{Casting de tipos en C++}	
			\subsection{Casting de tipos en C++}		
			\begin{itemize}
			
				\normalSizeItem { Castings normales }
				\begin{lstlisting}
				int number = 100;
				float height = (int)number; // C
				height = int(number); // C++
				number = static_cast<int>(3.14);
				// const_cast, dynamic_cast, reinterpret_cast
				\end{lstlisting}
				
				\normalSizeItem { Conversiones implícitas o explícitas }
				\begin{lstlisting}[basicstyle={\tiny\ttfamily}]
				struct Foo {
					// implicit conversion
					operator int() const { return 7; } 
					// explicit conversion
					explicit operator int*() const { return nullptr; }   
					explicit Foo(size_t elementsCount) { ... }
				...
				Foo x;
				int* q = x; // Error
				\end{lstlisting}
				
			\end{itemize}
		\end{frame}
		
		\begin{frame}[fragile]{Range-based for loop. I/O manipulators}	
			\subsection{Range-based for loop. I/O manipulators}		
			\begin{itemize}
			
				\normalSizeItem { Bucle for "moderno" para colecciones [C++11] }
				\begin{lstlisting}
				vector<string> names{"daniel", "manuel"};
				
				for (const auto& name: names) {
					cout << names << endl;
				}
				
				\end{lstlisting}
				
				\normalSizeItem { Mostrar \texttt{true}/\texttt{false} con \texttt{bool} y más precisión \texttt{float} }
				\begin{lstlisting}[basicstyle={\tiny\ttfamily}]
				#include <iomanip>
				
				boolalpha(cout); // o: cout.flags(std::ios_base::boolalpha);
				// "noboolalpha(cout)" para deshabilitarlo
				bool isHidden = false;
				cout << true << ' ' << isHidden << endl; // true false
				
				const long double pi = std::acos(-1.L);
				cout.precision(std::numeric_limits<long double>::digits10 + 1);
				cout << pi << endl; // 3.141592653589793239
				\end{lstlisting}
				
			\end{itemize}
		\end{frame}
		
		\begin{frame}[fragile]{\ttfamily{numeric\_limits} e \ttfamily{initializer\_list}}	
			\subsection{\ttfamily{numeric\_limits} e \ttfamily{initializer\_list}}		
			\begin{itemize}
			
				\normalSizeItem { Límites de valores }
				\begin{lstlisting}[basicstyle={\tiny\ttfamily}]
				// Antiguamente [<climits>]
				INT_MIN // -2147483648
				LONG_MAX // 9223372036854775807
				
				// Hoy en día [<cstdint>] y [<limits>]
				INT32_MIN // -2147483648
				UINT8_MAX // 255
				numeric_limits<uint16_t>::max() // 65535
				numeric_limits<float>::lowest() // -3.40282e+38
				
				\end{lstlisting}
				
				\normalSizeItem { initializer\_list }
				\begin{lstlisting}[basicstyle={\tiny\ttfamily}]
				class vector {
					vector(...) {}
					vector(initializer_list ilist) {}
				...
				
				vector numbers = {1,2,3,4}; // vector
				auto moreNumbers = {1,3,5,7}; // initializer_list
				\end{lstlisting}
				
			\end{itemize}
		\end{frame}
		
		\begin{frame}[fragile]{Valor aleatorio [C++11]}	
			\subsection{Valor aleatorio [C++11]}
			\begin{itemize}
			
				\normalSizeItem { Generar un número aleatorio entre un rango }
				\begin{lstlisting}
				#include <vector>
				#include <random>
				
				using namespace std;
				
				int main() {	
					
					vector<int> numbers {1,2,3,4,5,6};
					
					random_device randomDevice;
					mt19937 generator(randomDevice());
					
					uniform_int_distribution<int> randomValue(1, 90);
					cout << randomValue(generator) << endl;
				}
				\end{lstlisting}				
			\end{itemize}
		\end{frame}
		
		\begin{frame}[fragile]{Funciones de \ttfamily{<algorithm>}}	
			\begin{itemize}
			
				\normalSizeItem { Ordena, desordena, rellena, copia... }
				\begin{lstlisting}[basicstyle={\tiny\ttfamily}]
				vector<int> numbers {1,2,3,4,5,6};
	
				random_device randomDevice;
				mt19937 generator(randomDevice());
				
				// Posible salida: 6 4 1 3 2 5 
				std::shuffle(numbers.begin(), numbers.end(), generator);
				
				numbers = {1,2,3,4,5,6};
				
				vector<int> numbers2 {9,10};
				
				// Resultado numbers: {9,10,3,4,5,6}
				copy(numbers2.begin(), numbers2.end(), numbers.begin());
				
				// Resultado: 3 4 5 6 9 10
				sort(numbers.begin(), numbers.end());
				
				// ¿Y si quiero ordenarlo al revés...?
				\end{lstlisting}				
			\end{itemize}
		\end{frame}
		
	\newSectionWithoutIndex{Programación Funcional}	
	
		\begin{frame}[fragile]{[C++11] Expresiones lambda \lambda}	
			\subsection{[C++11] Expresiones lambda \lambda}		
			\begin{itemize}
			
				\normalSizeItem { Pasar funciones inline }
				\begin{lstlisting}
				// Resultado: 10 9 6 5 4 3 
				sort(numbers.begin(), numbers.end(), [](int lhs, int rhs) { 
					return lhs >= rhs; 
				});
				
				\end{lstlisting}	
				
				\normalSizeItem { Sintaxis }
				\begin{lstlisting}
				[ capture-list ] ( params ) -> ret { body }
				
				capture-list:
				[a,&b] - Captura "a" por copia, "b" por referencia
				[this] - Captura el valor del objeto actual
				[&] - Captura variables por referencia
				[=] - Captura variables por copia
				[] - No captura nada
				\end{lstlisting}	
							
			\end{itemize}
		\end{frame}
		
		\begin{frame}[fragile]{[C++11] Expresiones lambda \lambda}			
			\begin{itemize}				
				\normalSizeItem { ¿Cómo aceptar funciones? }
				\begin{lstlisting}[basicstyle={\tiny\ttfamily}]
				#include <functional>
				
				double operation(double lhs, double rhs, 
							std::function<double(double,double)> operationFunctor) {
					return operationFunctor(lhs, rhs);
				}
				
				int main() {
				
					double multiply = operation(10, 30.1, [](double lhs, double rhs) {
						return lhs * rhs;
					});
					
					double add = operation(20.1, 70, [](double lhs, double rhs) {
						return lhs + rhs;
					});
					
					double number = 1000;
					double custom = operation(20.1, 70, [&](double lhs, double rhs){
						return (number * lhs) + rhs;
					});
				}
				\end{lstlisting}
				
			\end{itemize}
		\end{frame}
		
		\begin{frame}[fragile]{Filter, Map \& Reduce (evt::Array)\smallCite{Array}}	
		
			\begin{itemize}
				\normalSizeItem { Filter }
				\begin{lstlisting}
				Array<string> names {"Daniel", "John", "Peter"};
				auto filtered = names.filter([](const string& str) {
					return str.size() > 4;
				}); // ["Daniel", "Peter"]
				\end{lstlisting}
				
				\normalSizeItem { Map \& Reduce }
				\begin{lstlisting}
				size_t totalSize = Array<string>({"names", "john"})
				    .map<size_t>([](auto str) {
					    return str.size();
				    }) // [5, 4]
				    .reduce<size_t>([](auto total, auto strSize) {
					    return total + strSize; 
				    }); // 9
				\end{lstlisting}
			\end{itemize}
		\end{frame}

		\newSectionWithoutIndex{Avanzado}
	
		\begin{frame}[fragile]{Expresiones constantes \ttfamily{(constexpr)\smallCite{constexpr}}}	
			\subsection{Expresiones constantes \ttfamily{(constexpr)}}		
			\begin{itemize}
			
				\normalSizeItem { Funciones }
				\begin{lstlisting}
				constexpr uint64_t factorial(int n) {
					return n <= 1 ? 1 : (n * factorial(n - 1));
				}
				
				int main {
					// error: static_assert failed "wrong!"
					static_assert(factorial(2) == 3, "wrong!");
				}
				
				\end{lstlisting}
				
				\normalSizeItem { Variables }
				\begin{lstlisting}
				constexpr uint64_t factorialResult = factorial(2);
				// static_assert(factorialResult == 3, "wrong!");
			
				constexpr uint64_t other = factorialResult * 2;
				static_assert(other == 4, "wrong!");
				\end{lstlisting}
				
			\end{itemize}
		\end{frame}
		
		\begin{frame}[fragile]{Expresiones constantes \ttfamily{(constexpr)}}		
			\begin{itemize}
			
				\normalSizeItem { Objetos }
				\begin{lstlisting}
				class Circle {
					int x_;
					int y_;
					int radius_;
				public:
					constexpr Circle (int x, int y, int radius): 
						x_(x), y_(y), radius_(radius) {}
					constexpr double area() const {
						return radius_ * radius_ * 3.1415926;
					}
					// ...
				};
				
				int main() {
					constexpr auto myCircle = Circle(10,20,30);
					static_assert(myCircle.area() < 3000, "wrong!");
				}
				
				\end{lstlisting}
			\end{itemize}
		\end{frame}
		
		\begin{frame}[fragile]{Expresiones constantes \ttfamily{(constexpr)}}	
			\begin{itemize}
				\normalSizeItem { Comprobaciones en tiempo de ejecución }
				\begin{lstlisting}
				#include <type_traits> // is_unsigned, is_object, etc
				
				template <typename Type>
				void doSomething(Type something) {	
					if constexpr (std::is_same<Type, int>()) {
						Type number = something * 10;
						cout << "int!!: " << number << endl;
					}
					else if constexpr (std::is_same<Type, string>()) {
						Type str = something;
						cout << "string!!: " << str.length() << endl;
					}
				}
	
				int main() {
					doSomething(800);
					doSomething("Daniel"s);
				}
				\end{lstlisting}
			\end{itemize}
		\end{frame}
		
		\begin{frame}[fragile]{Valor opcional}	
			\subsection{Valor opcional}
			\begin{itemize}
				\normalSizeItem { A veces no ``debemos'' devolver un valor concreto }
				\begin{lstlisting}[basicstyle={\tiny\ttfamily}]
				#include <iostream>
				#include <vector>
				#include <experimental/optional>
				
				using namespace std::experimental;
				
				size_t indexOf(int number, std::vector<int> numbers) {
					for (size_t i = 0; i < numbers.size(); i++) {
						if (number == numbers[i]) { 
							return i; 	
						}
					}
					return 0; // Confusión, ¿y si la Posición que devuelvo es 0? 
				}
				
				std::optional<size_t> safeIndexOf(int number, std::vector<int> numbers) {
					for (size_t i = 0; i < numbers.size(); i++) {
						if (number == numbers[i]) {
							return i;
						}
					}
					return nullopt;
				}
				\end{lstlisting}
			\end{itemize}
		\end{frame}
		
		\begin{frame}[fragile]{Valor opcional}	
			\begin{itemize}
				\normalSizeItem { La forma más segura es con el optional }
				\begin{lstlisting}
				int main() {
	
					vector<int> otherNumbers{};
				
					size_t index = indexOf(100, otherNumbers);
					// Segmentation fault
					// cout << otherNumbers[index] << endl;
					
					vector<int> numbers{1,2,3,4,10};
					if (auto index2 = safeIndexOf(100, numbers)) {
						cout << numbers[*index2] << endl;
						// index2.value_or(0)
					}
				}
				\end{lstlisting}
			\end{itemize}
		\end{frame}
		
		\begin{frame}[fragile]{Valor opcional}	
			\begin{itemize}
				\normalSizeItem { Valor que devuelve un vector al acceder a una posición }
				\begin{lstlisting}
				inline optional<Type> at(const size_t index) const {
					if (index >= count_) {
						return nullopt;
					} else {
						return this->values[index];
					}
				}
				\end{lstlisting}
			\end{itemize}
		\end{frame}
		
		\begin{frame}[fragile]{\ttfamily{std::enable\_if<>}}
			\subsection{\ttfamily{std::enable\_if<>}}	
			\begin{itemize}
				\normalSizeItem { Restringir tipos en Templates }
				\begin{lstlisting}
				#include <type_traits>
				
				template <typename ArithmeticType, 
					typename = typename std::enable_if<
						std::is_arithmetic<ArithmeticType>::value
					>::type>
				ArithmeticType doSomething(ArithmeticType number) {
					return number * 100;
				}
				
				int main() {
				
					doSomething(800); // OK
					
					// candidate template ignored: disabled by 'enable_if'
					doSomething("Daniel"s); // ERROR 
				}
				\end{lstlisting}
			\end{itemize}
		\end{frame}
		
		\begin{frame}[fragile]{Buenas prácticas y consejos para clases \smallCite{cppbestpractices}}	
			\subsection{Buenas prácticas y consejos para clases}
			\begin{itemize}
				\normalSizeItem { include/Human.hpp }
				\begin{lstlisting}[basicstyle={\tiny\ttfamily}]
				#pragma once // en vez de los clásicos guards (#ifndef...)
				
				#include <cstdint>
				#include <string_view> // C++17
				// using namespace std; // No usar en ficheros .h, .hpp

				namespace evt { // MUY recomendable
					class Human {
						// Inicialización de variables directamente [c++11]
						// NUNCA usar guiones bajos al principio de nombres
						uint8_t age_{};
						std::string_view name_{};
					public:
						// constexpr Human(){} // Podría o no tener sentido
						constexpr Human(const uint8_t age, const std::string_view& name): age_(age), name_(name) {}
						
						constexpr uint8_t age() const {  
							return this->age_;
						}
						constexpr std::string_view name() const { 
							return this->name_;
						}
					};
				}
				\end{lstlisting}
			\end{itemize}
		\end{frame}
		
		\begin{frame}[fragile]{Buenas prácticas y consejos para clases}	
			\begin{itemize}
				\normalSizeItem { Main.cpp }
				\begin{lstlisting}[basicstyle={\tiny\ttfamily}]
				#include "include/Human.hpp"
				
				// using namespace evt; (opcional)
				
				// No es recomendable pasar datos complejos por copia...
				// void giveMeTheObject(evt::Human human) { ... }
				
				// ... ni tampoco solo por referencia (se podrían editar los originales)
				// void giveMeTheObject(evt::Human& human) { ... }
				
				// Mejor por referencia constante
				void giveMeTheObject(const evt::Human& human) { ... }
				
				int main() {
					constexpr evt::Human daniel(10, "Daniel");
					static_assert(daniel.name() == "Daniel", "incorrect!");
				}
				\end{lstlisting}
			\end{itemize}
		\end{frame}
		
	\newSectionWithoutIndex{Compilación}	
  
		\begin{frame}[fragile]{Compilar con nuevas versiones}	
			\subsection{Compilar con C++11}
			\begin{itemize}
			
				\normalSizeItem {Añadir la versión de C++ tras \texttt{-std=}}
				\begin{lstlisting}
				g++ main.cpp -std=c++11
				g++ main.cpp -std=c++14
				g++ main.cpp -std=c++17 // o: 1z
				\end{lstlisting}
				
				\normalSizeItem {\textit{Flags} recomendados}
				\begin{lstlisting}
				-Wall ->	  Habilita la mayoría de warnings
				-Wextra ->	Habilita más warnings
				-O1 / -O2 / -Os / -O3 / -fast ->	Optimizaciones
				~~
				g++ main.cpp -std=c++14 -Wall -Wextra -Os
				\end{lstlisting}
				
			\end{itemize}
		\end{frame}
		
		\begin{frame}[fragile]{\texttt{Makefile} de ejemplo}	
			\subsection{\texttt{Makefile} de ejemplo}
			\begin{itemize}
			
				\normalSizeItem {Ejemplo completo}
				\begin{lstlisting}
				## FLAGS ##
				Libraries = -L lib
				Headers = -I include
				Sources = main.cpp $(Headers) $(Libraries)
				CompilerFlags = -std=c++14 -Os -Wall -Wextra
				OutputName = test
				
				## TARGETS ##
				all: 
					@g++ $(CompilerFlags) $(Sources) -o $(OutputName)
				
				clean:
					@rm -i $(OutputName)*
				\end{lstlisting}
				
			\end{itemize}
		\end{frame}
		
	\newSectionWithoutIndex{Bibliografía}	
		\tiny
		\begin{frame}[allowframebreaks]{Bibliografía}
			\bibliographystyle{plainurl}
			\bibliography{References}
			\small
			Recomendaciones:
			\begin{itemize}
				\urlItem{http://devdocs.io}
				\urlItem{https://tex.stackexchange.com}
			\end{itemize}
		\end{frame}
\end{document}
